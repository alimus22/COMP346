\newpage
\subsection*{Question 2 - Part B}

\noindent In part B, we use the selenium extension for Google Chrome to test the UI of a google search result page. Selenium let you record the UI and the different interaction the user makes with it, and then add \textit{"commands"} that let you verify certain information about the web page that has been record, for example verifies if a certain element is present or not. \\ We had to verify that the elements \textit{Images, Shopping, Videos, and News} are present on the search result page on Google. To do that, we start a new Selenium project, start create a new test, start the recording on Google's homepage, search something in Google (i.e. "arbitrary search"), once the search result page has loaded, we can stop the recording. From there, we must add commands to verify that all the elements we must check are present. To do so we use the \textit{assert element present} command to verify that each element is present. When it's done, we can export the test (named testcase2) and save the Selenium project (which returns a .side file). Adding the test \verb|testcase2()| in the \verb|DemoTutorialTest| class will add the test to our existing maven project. Note that the project used in part A was copied and reused for part B. \\ The whole project, including the .side file and the \verb|testCase2.java| file generated by selenium, are joined with this submission. 